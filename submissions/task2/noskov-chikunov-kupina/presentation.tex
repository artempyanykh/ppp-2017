\documentclass{beamer}
%\usepackage[left=2cm,right=2cm,top=2cm,bottom=2cm,bindingoffset=0cm]{geometry}

\usepackage[english,russian]{babel}
\usepackage[utf8]{inputenc}
\usetheme{Warsaw}
\usepackage[warn]{mathtext}				% Поддержка русского текста в формулах
\usepackage[T1, T2A]{fontenc}			% Пакет выбора кодировки и шрифтов


\begin{document}

\title{Пакеты прикладных программ, задание 2}
\author{Купина М., Носков Г., Чикунов М.}
\section{Инструкция по запуску:}

% Автоматическая генерация содержания
\frame{\titlepage}

\begin{frame}{Инструкция по запуску:}
1)Указать путь до файла production-data.csv

2)Запустить программу.

\end{frame}
\section{Решение задачи:}

\begin{frame}{Решение задачи:}
Любая война является экономической игрой, поэтому важно, чтобы армия нашей королевы, Дейенерис из дома
Таргариенов, именуемая первой, Неопалимая, Королева Миэрина, Королева Андалов, Ройнар и Первых Людей, Кхалиси
Дотракийского Моря, Разбивающая Оковы и Матерь Драконов, была постоянно снабжена \textbf{качественным} вооружением. Советник королевы Тирион Ланнистер сделал правильный выбор, что обратился к нам с просьбой о помощи.
\end{frame}

\begin{frame}{Решение задачи:}
Каждый солдат в армии должен быть вооружен, поэтому для начала оценим сколько произвели вооружения каждый из поставщиков.
\begin{figure}
\begin{center}
\includegraphics[width=0.5\textwidth]{all_produced.jpg}
\end{center}
\end{figure}
Как оказалось, оба поставщика производят практически одинаковое кол-во оружия, поэтому этот критерий рассматривать не будем.

\end{frame}

\begin{frame}{Решение задачи:}
Так как поставщики производят практически одинаковое кол-во товара, то нам остается оценить бракованный товар.
Для этого рассчитаем средний \% брака каждого поставщика на iый месяц после производства.
\begin{figure}
\begin{center}
\includegraphics[width=0.8\textwidth]{def_trend.jpg}
\end{center}
\end{figure}
\end{frame}
\begin{frame}{Решение задачи:}
Проанализировав график, понятно, что у поставщика \\ Westeros Inc. основной брак выявляется в первые месяца и далее убывает, в это время у Harpy Co., наоборот, с каждым месяцем \% брака растет. 
Уже можно предположить, что заключить контракт с поставщиком Westeros Inc. нам будет выгоднее, но не будем спешить и посмотрим на суммарный рост относительного дефекта.

\end{frame}

\begin{frame}{Решение задачи:}

\begin{figure}[htb]
\center{\includegraphics[width=0.8\linewidth]{cum_def_1.jpg} }
\end{figure}
\end{frame}

\begin{frame}{Решение задачи:}

\begin{figure}[htb]
\center{\includegraphics[width=0.7\linewidth]{cum_def_2.jpg} }
\end{figure}
Как мы и предполагали, у поставщика Westeros Inc. в первые 2 месяца наибольший \% дефекта.
Но уже можно заметить, как растет дисперсия брака.
\end{frame}


\begin{frame}{Решение задачи:}

\begin{figure}[htb]
\flushleft{Посмотрим на 5ый месяц:}
\center{\includegraphics[width=0.7\linewidth]{cum_def_5.jpg} }
\end{figure}
Видно, как у Harpy Co. \% брака уже выше, чем у Westeros Inc.
\end{frame}



\begin{frame}{Решение задачи:}

\begin{figure}[htb]
\flushleft{И, наконец, 6ой месяц:}
\center{\includegraphics[width=0.7\linewidth]{cum_def_6.jpg} }
\end{figure}
Невооруженным взлядом видно, что брака у Harpy Co. больше.
Оценим изменение брака за все 7 месяцев.
\end{frame}

\begin{frame}{Решение задачи:}

\begin{figure}[htb]
\center{\includegraphics[width=0.6\linewidth]{cum_def_trend.jpg} }
\end{figure}
Наши предположения подтвердились, примерно с 7 месяца в среднем суммарный \% брака у Westeros Inc. будет меньше,
поэтому в течение 11 месяцев, пока идет война с 7ю королевствами, нам будет выгодно сотрудничать с \\
\textbf{Westeros Inc.} 

\end{frame}

\end{document}
