\documentclass{beamer}

\usepackage[T2A]{fontenc}
\usepackage[utf8]{inputenc}
\usepackage[english,russian]{babel}
\usepackage{amssymb,amsfonts,float,mathtext}
\usepackage{cite,enumerate,float,indentfirst}
\usepackage{graphicx}


\title{Задание 2}
\author{Соболькова E., Запорожец А., Агаджанян Е.}
\institute{МГУ имени М. В. Ломоносова}
\date{Москва, 2017}
% выбор темы

\begin{document}
	\maketitle
	
\begin{frame}{Описание задания} 
	После оглушительного успеха в освобождении Астапора, Миэрина и Юнкая от власти работорговцев Дейенерис Бурерожденная открыла себе доступ к Летнему морю, а следовательно — путь в Вестерос. 
	Для ведения войны с Семью Королевствами нужно оружие, а для оружия нужна сталь. Нет никаких сомнений в кузнечном искусстве Безупречных, однако поставщики стали не столь надежны. 
	Два основных поставщика стали — это Westeros Inc. и Harpy \& Co. 
\end{frame} 

\begin{frame}{Описание задания} 
На протяжении нескольких месяцев мы закупаем сталь у обеих компаний, и каждая из них предлагает ощутимую скидку при заключении эксклюзивного договора на поставку.
Советник королевы Тирион Ланнистер знает о твоем умении принимать взвешенные рациональные решения и просит помощи в объективном решении вопроса о том, с какой из компаний следует заключить эксклюзивный договор на поставку стали. 
У Тириона есть записи о производстве мечей каждым из кузнецов-безупречных, а также данные о количестве сломанных мечей в каждый из месяцев ведения боевых действий. 
\end{frame} 

\begin{frame}{Выполнение задания} 
Необходимо провести разведывательный анализ данных с целью ответа на вопрос: "С каким из поставщиков стали следует заключить договор?"
\end{frame}
% 1 слайд
\begin{frame}
Для того, чтобы определить, к какой компании стоит обратиться, чтобы производить оружие, проанализируем статистику по дефектам. Для этого  возьмем процент дефектных орудий для каждой произведенной партии. Таким образом, получим две таблицы данных для компаний Harpy&co и Westeros.inc. 
\end{frame}

% 2 слайд 
\begin{frame}
Воспользуемся диаграммой размаха для наглядного представления данных с с помощью функции Boxplot().
Для удобства, отобразим наши данные на единой картинке, где желтый цвет соответствует компании Westeros.inc, а зеленый - Harpy & co.
\end{frame}

% 3 слайд с картинкой 
\begin{frame}

\begin{figure}[t]
	\centering
	\includegraphics[width=0.7\textwidth]{2}
\end{figure}

\end{frame}

% 4 слайд
\begin{frame}
Если посмотреть на первые три месяца, то у Westeros.inc процент сломанных мечей выше, чем у Harpy. Однако, начиная с четвертого месяца у компании Harpy процент ломающихся мечей резко возрастает. В компании Westeros начиная с четвертого месяца мы наблюдаем явную тенденцию снижения процента сломанных мечей. 
\end{frame}

%5 слайд
\begin{frame}
Построим регрессионную модель для анализа и предсказания дефектов на будущие 11 месяцев. Рассчитаем среднее количество брака на i-ый месяц. Желтый цвет соответствует компании Westeros.inc, а зеленый - Harpy & co.
\end{frame}

%6 слайд картинка
\begin{frame}

\begin{figure}[t]
\centering
\includegraphics[width=0.9\textwidth]{5}
\end{figure}
\end{frame}

%7 слайд
\begin{frame}
Изучив построенный график можно сделать вывод, что основной брак компании Westeros приходится на первые месяцы. Среднее значение показывает линейную динамику, линейная регрессия дает нам основание утверждать, что компания Westeros более предпочтительна для сотрудничества, чем компания Harpy.
\end{frame}

% 8 слайд итог
\begin{frame}
Обе компании имеют свои плюсы и минусы. Компания Harpy имеет отличные показатели в первые три месяца. Однако, мы планируем закупать оружие в течение 11 месяцев. На этом длительном промежутке компаниия Westeros более выигрышна, так как имеет линейную динамику. В силу проведенного анализа,  у компании Harpy заказывать производство мечей более рискованно, так как вероятность поломки постоянно меняется. Поэтому свое предпочтение мы отдаем Westeros.
\end{frame}

% 9 слайд
\begin{frame}



\end{frame}
\begin{frame}{Задание выполняли} 
\begin{itemize} 
	{ \item Запорожец Анастасия, студентка 412 группы. 
		
		Занималась написанием программы.
		\item Соболькова Екатерина, студентка 412 группы. Анализировала данные и писала отчет.
		\item Агаджанян Елизавета, студентка 412 группы. 
		
		Составляла презентацию и писала отчет. 
	} 
\end{itemize}
\end{frame}

\end{document}
