\documentclass{beamer}

\usepackage[T2A]{fontenc}
\usepackage[utf8]{inputenc}
\usepackage[english,russian]{babel}
\usepackage{amssymb,amsfonts,float,mathtext}
\usepackage{cite,enumerate,float,indentfirst}
\usepackage{graphicx}


\title{Задание 2}
\author{Соболькова E., Запорожец А., Агаджанян Е.}
\institute{МГУ имени М. В. Ломоносова}
\date{Москва, 2017}
% выбор темы

\begin{document}
	\maketitle
	
\begin{frame}{Описание задания} 
	После оглушительного успеха в освобождении Астапора, Миэрина и Юнкая от власти работорговцев Дейенерис Бурерожденная открыла себе доступ к Летнему морю, а следовательно — путь в Вестерос. 
	Для ведения войны с Семью Королевствами нужно оружие, а для оружия нужна сталь. Нет никаких сомнений в кузнечном искусстве Безупречных, однако поставщики стали не столь надежны. 
	Два основных поставщика стали — это Westeros Inc. и Harpy \& Co. 
\end{frame} 

\begin{frame}{Описание задания} 
На протяжении нескольких месяцев мы закупаем сталь у обеих компаний, и каждая из них предлагает ощутимую скидку при заключении эксклюзивного договора на поставку.
Советник королевы Тирион Ланнистер знает о твоем умении принимать взвешенные рациональные решения и просит помощи в объективном решении вопроса о том, с какой из компаний следует заключить эксклюзивный договор на поставку стали. 
У Тириона есть записи о производстве мечей каждым из кузнецов-безупречных, а также данные о количестве сломанных мечей в каждый из месяцев ведения боевых действий. 
\end{frame} 

\begin{frame}{Выполнение задания} 
Необходимо провести разведывательный анализ данных с целью ответа на вопрос: "С каким из поставщиков стали следует заключить договор?"
\end{frame}
% 1 слайд
\begin{frame}
\textbf{BoxPlot}

Проанализируем наши данные с помощью диаграммы размаха(ящик с усами), с помощью функции Boxplot(). Для этого сравним такие характеристики как Defects и Produced для двух компаний, которые занимаются производством оружия harpy.co и westeros.inc.

По данной диаграмме( представленной на следующем слайде) видно, что в основном количество дефектов у оружия, произведенного компанией harpy.co значительно меньше, чем westeros.inc.Так как медиана в первом случае ниже, чем медиана во втором случае. Однако, стоит заметить, что для компании westeros.inc нет характерных выбросов. А вот в harpy.co мы видим целых 12. Это говорит о том, что в какие-то единичные моменты может появляться большое количество дефектов.
\end{frame}

% 2 слайд с картинкой для BoxPlot
\begin{frame}
\begin{figure}[t]
	\centering
	\includegraphics[width=0.95\textwidth]{1}
\end{figure}
\end{frame}

% 3 слайд со второй картинкой и продолжением BoxPlot
\begin{frame}
Следующая диаграмма нам показывает, что либо не производится ни одного оружия, либо они производятся уже в большем количестве единоразово. В данном случае от 100. Это верно для обеих компаний.
\begin{figure}[t]
	\centering
	\includegraphics[width=0.7\textwidth]{2}
\end{figure}

\end{frame}

% 4 слайд: Текст для Histogram с картинкой
\begin{frame}
\textbf{Histogram}

Проанализируем наши данные с помощью гистограммы. В этом нам поможет функция hist(). В результате выполнения этой команды, будут посчитаны частоты появления различных значений элементов передаваемого вектора.
\begin{figure}[t]
\centering
\includegraphics[width=0.47\textwidth]{3}
\end{figure}
\end{frame}

%5 слайд: текст + график
\begin{frame}
Заметим, что эти данные были построены сразу для двух компаний, производящих оружие. Для того, чтобы провести более качественный анализ для каждой из компаний, воспользуемся функцией qplot().

Теперь мы видим, какое количество дефектов и сколько раз было у компании happy.co (персиковый цвет) и компании western.inc (мятный цвет).
\begin{figure}[t]
\centering
\includegraphics[width=0.6\textwidth]{4}
\end{figure}
\end{frame}

%6 слайд: текст + картинка( продолжение по смыслу)
\begin{frame}
Для большей наглядности представим данные в виде
\begin{figure}[t]
\centering
\includegraphics[width=0.9\textwidth]{5}
\end{figure}
\end{frame}

%7 слайд: Анализ Histogram
\begin{frame}
Основываясь на данном представлении мы можем сделать вывод, что в компании harpy.co очень много дефектов, которые принимают небольшие значения (примерно от 0 до 5), однако есть достаточно значимое количество дефектов с большими значениями(от 10 до 20), которыми нельзя пренебречь или сказать, что это случайные выбросы, так как частота их встречаемости ощутима. В компании westeros.inc основные значения дефектов колеблются от 4 до 12, что является весомым минусом.
\end{frame}

% 8 слайд: Scatter Plot текст
\begin{frame}
\textbf{Scatter Plot}

Посмотрим на график рассеивания для двух компаний harpy.co и Westeros.inc. Красным цветом изображено какое количество оружия в месяц было произведено в соответствующей компании, зеленым - количество сломанных мечей в данном месяце. Существенных различий между данными нету. Однако, видно, что в 3 месяце компания Westeros.inc произвела оружия меньше, чем в этом же месяце компания Harpy.co.
\end{frame}

% 9 слайд: Картинка Scatter Plot
\begin{frame}
\begin{figure}[t]
\centering
\includegraphics[width=1\textwidth]{6}
\end{figure}
\end{frame}

% 10 слайд: Pareto plot
\begin{frame}
\textbf{Pareto plot}


Построим диаграмму Парето для двух компаний Westeros.inc и harpy.co
\begin{figure}[t]
\centering
\includegraphics[width=0.9\textwidth]{7}
\end{figure}
\end{frame}

% 11 слайд: Pareto plot
\begin{frame}
\begin{figure}[t]
	\centering
	\includegraphics[width=0.9\textwidth]{8}
\end{figure}
\end{frame}


%12 слайд: анализ картинки
\begin{frame}
Для начала посмотрим статистику по производству оружия для этих компаний. По оси OX мы видим количество произведенного оружия, слева по оси OY - сколько раз встречалось данное наблюдение в нашей выборке, справа по оси OY - то же самое, только в процентном соотношении. Столбчатая диаграмма показывает нам распределение величин по всей выборке, а линия - суммирует предыдущие значения с текущим. Графики нам более наглядно представляют данные. Обратимся к числам, стоящим ниже для более подробного анализа. 
Как мы видим, количество раз производства оружия у этих компаний одинаково - 200(1350-1050). Для компании harpy.co по 3\% составляют производство 101, 102, 104 деталей, по 2\% - 103, 105, 106, 107, 108, 109, 110. Для компании Westeros.inc по 3\% - 103, 104, 109, по 2\% - 101, 102, 105, 106, 107, 108, 110. В общей сложности компании произвели примерно равное количество оружия.

\end{frame}

% 13 слайд: статистика по дефектам ( картинка)
\begin{frame}
Посмотрим на статистику по дефектам:
\begin{figure}[t]
\centering
\includegraphics[width=1\textwidth]{9}
\end{figure}
\end{frame}

% 14 слайд: картинка
\begin{frame}
\begin{figure}[t]
\centering
\includegraphics[width=1\textwidth]{10}
\end{figure}
\end{frame}

%15 слайд: анализ статистики
\begin{frame}
По данной статистике мы видим, что в компании Westeros.inc достаточно часто встречаются поломки в объеме от 5 до 11, наибольшее число неисправностей приходятся на 7, 8 и 9. Гораздо в меньшей степени встречаются от 2 до 4, однако есть 12 - 2\% от общей выборки и 13 - 1\% от общей выборки. Все остальные не столь значительны.

Если же посмотреть на статистику Harpy\&co, то мы увидим, что значительная часть всех поломок приходится на объем от 2 до 4, что гораздо меньше, чем в Westeros.inc, однако мы видим, что пусть и не столь часто, но встречается большое число поломок: 14 - 4\% всех поломок; 16, 13, 15 - 3\%; 11, 12 - 2\%, 17, 10, 18 - 1\%. 
Таким образом, в компании Westeros.inc со средней периодичностью встречается среднее число дефектов(5-11), а в компании Harpy\&co часто встречается небольшое число дефектов(1-4), но есть выбросы с количеством дефектов в размере от 12 до 18.

\end{frame}

%16 слайд: Stem-and-leaf
\begin{frame}
\textbf{Stem-and-leaf}

Воспользуемся описательной статистикой - диаграммой «Ветвей и листьев». 
Длина каждой строки соответствует количеству наблюдений, попадающих в определенный интервал. Кроме того здесь также отображено численное значение для каждого наблюдения. Для этой цели численное значения разбиваются на два компонента: ветвь, представляющую собой десятки и лист — единицы. Ветвь соответствует тем разрядам численного значения наблюдаемой переменной, которые не изменяются, а листья — разрядам, которые изменяются в пределах избранного интервала.
\end{frame}

%17 слайд: две картинки
\begin{frame}
Посмотрим на результаты по производству оружия:
\begin{figure}[t]
\centering
\includegraphics[width=0.7\textwidth]{11}
\end{figure}

\begin{figure}[t]
\centering
\includegraphics[width=0.7\textwidth]{12}
\end{figure}
\end{frame}

%18 слайд
\begin{frame}
По данной диаграмме мы видим, что обе компании либо ничего не производят(строка из нулей), либо производят 101,102…110 оружия.

Теперь посмотрим на результаты дефектов оружия:

Первая диаграмма соответствует количеству дефектов для компании Westeros.inc, вторая - количеству дефектов для компании Harpy\&Co.
\end{frame}

%19 слайд
\begin{frame}
\begin{figure}[t]
	\centering
	\includegraphics[width=0.6\textwidth]{13}
\end{figure}

\begin{figure}[t]
	\centering
	\includegraphics[width=0.6\textwidth]{14}
\end{figure}
\end{frame}

% 20 слайд
\begin{frame}
Судя по первой диаграмме, в компании Westeros.inc оружие либо не ломается, либо поломки происходят достаточно часто и их количество колеблется от 3 до 11-12. Числа достаточно весомые, что не очень хорошо говорит качестве продукции.
Во второй диаграмме мы наблюдаем, что поломки делятся на 3 типа - либо совсем не ломаются, либо небольшое количество, но ломаются часто, либо большое количество дефектов(10-18), но случается относительно редко.

\end{frame}

% 21 слайд
\begin{frame}
\textbf{Parallel Coordinares}


На графике параллельных координат отображается множество рядов данных в виде линий, проходящих по промежуточным осям. Каждая из осей отображает значения по выбранному показателю. Линии соединяют точки на промежуточных осях в соответствии со значениями элементов по нашим показателям.

\end{frame}

% 22
\begin{frame}
\begin{figure}[t]
	\centering
	\includegraphics[width=1\textwidth]{15}
\end{figure}
\end{frame}

% 23 
\begin{frame}
По данному графику мы видим, что для переменной production.date характерно выделены 6 точек. Каждая точка соответствует месяцу, в который было произведено оружие. Следующие 7 точек это месяцы, в которые пришел отчет о дефектах. Нумерация идет снизу. Можно сделать вывод, что по первому месяцу отчет приходит в 1 и 2 … 7 месяцы, по второму - во 2, 3 … 7 и тд.  Для переменной produced у нас какрактерно выделен 0 и несколько точек, соответствующих значениям от 101 до 120, что говорит нам о том, что обе компании либо производят 0, либо сразу большое количество. Далее идет приличные разброс по дефектам для каждой компании и завершение на самих компаниях. 
\end{frame}

%24
\begin{frame}
\textbf{PCA}


Прежде чем применять метод главных компонент, взглянем на наши данные:
\begin{figure}[t]
	\centering
	\includegraphics[width=0.7\textwidth]{16}
\end{figure}
\begin{figure}[t]
	\centering
	\includegraphics[width=0.7\textwidth]{17}
\end{figure}
\end{frame}

%25
\begin{frame}
Средние значения сильно разнятся между собой. Соответственно прежде чем применять метод главных компонент, необходимо стандартизировать переменные. 

Посмотрим описание результатов оценивания главных компонент:
\begin{figure}[h]
	\centering
	\includegraphics[width=0.7\textwidth]{18}
\end{figure}
Для компании Westeros.inc
\end{frame}

% 26
\begin{frame}
Посмотрим, как главные компоненты справляются с описанием наших данных и на сколько мы можем уменьшить нашу размерность, если изначально она равна 5(unsullen.id, production.date, report.date, produced, defects) PC1 описывает примерно 41\% всех данных, PC2 - 28\%, PC3 - 20\%, PC4 - 8\%, PCA5 - 1\%. В сумме PC1 PC2 PC3 дают нам информацию о 89,9\% информации. На практике берут те компоненты, которые содержат около 88\% информации. В нашем случае это первые три компоненты.
\end{frame}

%27
\begin{frame}
\begin{figure}[t]
	\centering
	\includegraphics[width=1\textwidth]{19}
\end{figure}
Для компании Harpy\&Co
\end{frame}

%28
\begin{frame}
Посмотрим, как главные компоненты справляются с описанием наших данных и на сколько мы можем уменьшить нашу размерность, если изначально она равна 5(unsullen.id, production.date, report.date, produced, defects) PC1 описывает примерно 38\% всех данных, PC2 - 28\%, PC3 - 20\%, PC4 - 10\%, PCA5 - 1\%. В сумме PC1 PC2 PC3 дают нам информацию о 87,1\% информации. На практике берут те компоненты, которые содержат около 88\% информации. В нашем случае это первые три компоненты.
Итак, для обеих компаний достаточно первых трех главных компонент, чтобы описать основные данные. Для компании Westeros.inc мы смогли описать на 2\% лучше.
\end{frame}

%29
\begin{frame}
Посмотрим на график объяснений дисперсии:
\begin{figure}[t]
	\centering
	\includegraphics[width=0.5\textwidth]{20}
\end{figure}
\begin{figure}[t]
	\centering
	\includegraphics[width=0.5\textwidth]{21}
\end{figure}
\end{frame}

%30
\begin{frame}

Визуализируем наши данные для более наглядного представления. Построим график от первых двух компонент. По оси OX будет PC1, по оси OY будет PC2

Красные векторы это отмеренные исходные переменные в координатах первых двух главных компонент. Соответственно, проецируя такой вектор на одну из осей, мы получим с каким весом она входит в соответствующую компоненту.
\end{frame}

% 31
\begin{frame}
\begin{figure}[t]
	\centering
	\includegraphics[width=0.9\textwidth]{22}
\end{figure}
\end{frame}

% 32
\begin{frame}
\begin{figure}[t]
	\centering
	\includegraphics[width=0.9\textwidth]{23}
\end{figure}
\end{frame}

% 30 слайд
\begin{frame}
Для компании Westeros.inc:
Мы видим, что вторая главная компонента с большим весом себе берет production.date и report.date, а первая главная компонента produced и defects. 
Анализируя полученные результаты, можно сказать, что первая главная компонента у нас отвечает за производство и дефекты, а вторая за время производства и получение отчетов.
Для компании Harpy\&co:
Мы видим, что вторая главная компонента с большим весом себе берет production.date и report.date, а первая главная компонента produced и defects.
\end{frame}

% 31 слайд: анализ
\begin{frame}
Анализируя полученные результаты, можно сказать, что первая главная компонента у нас отвечает за производство и дефекты, а вторая за время производства и получение отчетов.  
Таким образом, отличие между Westeros.inc и Hsrpy\&co заключается лишь в том, что production.date и report.date смотрят в разные стороны(то есть противоположны по знаку), хотя проекции по модулю на вторую главную компоненту очень похожи между собой.
\end{frame}

%32
\begin{frame}
Для большей наглядности посмотрим численные характеристики:
С какими коэффициентами входят исходные переменные в первую компоненту для компании Westeros.inc (w.v1) и Harpy\&co (h.v1):
\begin{figure}[t]
	\centering
	\includegraphics[width=1\textwidth]{24}
\end{figure}
\end{frame}

\begin{frame}
С какими коэффициентами входят исходные переменные в первую компоненту для компании Westeros.inc (w.v2) и Harpy\&co (h.v2):
\begin{figure}[t]
	\centering
	\includegraphics[width=1\textwidth]{25}
\end{figure}
\end{frame}

% 33 слайд
\begin{frame}
\textbf{Статистика по дефектам для компаний Westeros.inc и Harpy\&co и Итог}

Посмотрим, как распределены дефекты по месяцам в наших компаниях:
\begin{figure}[t]
	\centering
	\includegraphics[width=0.9\textwidth]{26}
	\label{fig:}
\end{figure}
\end{frame}

\begin{frame}
\begin{figure}[t]
	\centering
	\includegraphics[width=1\textwidth]{27}
\end{figure}
\end{frame}

% 36 слайд
\begin{frame}

Для начала посмотрим на первые три месяца. В обеих компаниях показатели по дефектам очень большие, однако в Westeros.inc выше 15 поломок нет. В компании Harpy\&co мы наблюдаем либо небольшие показатели в пределах 5, либо очень весомые дефекты вплоть до 20. Если же посмотреть на последующие 3 месяца, то в Westeros.inc регулярно встречаются поломки, которые колеблются от 5 до 12. В Harpy\&co количество дефектов почти сведено на нет, все показатели в пределах 5.


Основываясь на результатах проведенного выше анализа, мы делаем вывод, что в будущем лучше сотрудничать с компанией Harpy\&co.

\end{frame}
\begin{frame}{Задание выполняли} 
\begin{itemize} 
	{ \item Запорожец Анастасия, студентка 412 группы. 
		
		Занималась написанием программы.
		\item Соболькова Екатерина, студентка 412 группы. Анализировала данные и писала отчет.
		\item Агаджанян Елизавета, студентка 412 группы. 
		
		Составляла презентацию и писала отчет. 
	} 
\end{itemize}
\end{frame}

\end{document}