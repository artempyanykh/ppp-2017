\documentclass{beamer}
\usepackage[T2A]{fontenc}
\usepackage[cp1251]{inputenc}
\usepackage[english,russian]{babel}
\usepackage{amssymb,amsfonts,amsmath,mathtext}
\usepackage{cite,enumerate,float,indentfirst}

\usetheme{Warsaw}

\title{Итоги работы второго задания}
\author{Алияров Р., Байрамкулов А., Полушкин А.}
\institute{МГУ имени М. В. Ломоносова}
\date{Москва, 2017}

\begin{document}

%%титульная страница
\maketitle

\begin{frame}
\frametitle{Основные положения разведовательного анализа данных}
\begin{flushleft}
\small{Разведовательный анализ данных направлен на выявление основных характеристик и суммирование их с целью построения некоторых гипотез. Для удобства чаще всего используют визуализацию, по которой достаточно просто построить гипотезы.}
\end{flushleft}
\begin{center}
\begin{block}{Методы, используемые для разведовательного анализа данных}
\begin{itemize}
\item Использование пространственно-некогерентного света ртутной лампы;
\item Применение RAW-конверторов, использование полного динамического диапазона цифровой фотокамеры;
\item Усовершенствование линейных алгоримтов деконволюции;
\item Многовариантные графики;
\end{itemize}
\end{block}
\end{center}
\end{frame}

\begin{frame}
\begin{flushleft}
\frametitle{Этапы анализа}
\framesubtitle{Первый этап - анализ первичных данных}
\small{Рассмотрим данные по производству:}
\newline
\includegraphics[width=1\linewidth]{image/1.png}
\newline
\small{Harpy:}
\includegraphics[width=1\linewidth]{image/2.png}
\newline
\small{Westeros:}
\includegraphics[width=1\linewidth]{image/3.png}
\end{flushleft}
\end{frame}

\begin{frame}
\scriptsize{Все дефекты распределены по партиям, произведенным каждым кузнецом. Рассмотрим распределение дефектов по каждому кузнецу на первый месяц после производства, на второй и так далее.}
\includegraphics[width=1\linewidth]{image/4.png}
\newline
\scriptsize{Как видно из графиков harpy имеет большую дисперсию, а среднее имеет схожие черты с левой частью графика функции плотности нормального распределения.

График westeros имеет меньшую дисперсию, а среднее похоже на убывающую линейную функцию y = kx+b.}
\end{frame}

\begin{frame}
\begin{flushleft}
\frametitle{Этапы анализа}
\framesubtitle{Второй этап - оценки для анализа поставок стали каждой компанией}
\footnotesize{1. Построим график среднего для полученных выше распределений:}
\includegraphics[width=1\linewidth]{image/5.png}
\end{flushleft}
\end{frame}

\begin{frame}
\begin{flushleft}
\footnotesize{2. Добавим оценку математического ожидания для полученого выше среднего:}
\includegraphics[width=1\linewidth]{image/6.png}
\newline
\tiny{Из графиков видно, что оценка математического ожидания дефектов для компании Harpy незначительно выше, чем для компании Westeros:}
\includegraphics[width=1\linewidth]{image/7.png}
\end{flushleft}
\end{frame}

\begin{frame}
\begin{flushleft}
\footnotesize{3. Рассмотрим в качестве оценки показатель “выборочная дисперсия” - это оценка теоретической дисперсии распределения, рассчитанная на основе данных выборки. Для сохранения размерности, возьмем корень из дисперсии, то есть среднеквадратичное отклонение.}
\includegraphics[width=1\linewidth]{image/8.png}
\end{flushleft}
\end{frame}

\begin{frame}
\begin{flushleft}
\begin{block}{Выборочная дисперсия}
\tiny{Пусть}
\begin{math}
X_1,\ldots,X_n,\ldots
\end{math}
\tiny{ - выборка из распределения вероятности. Тогда}
\begin{itemize}
\item{
\tiny{выборочная дисперсия — это случайная величина }
\begin{math}
S^2_n=\frac{1}{n}\sum\limits_{i=1}^n\left(X_i-\bar{X}\right)^2=\frac{1}{n}\sum\limits_{i=1}^nX_i^2-\left(\frac{1}{n}\sum\limits_{i=1}^nX_i\right)^2
\end{math}
\tiny{где символ }
\begin{math}
\bar{X}
\end{math}
\tiny{обозначает выборочное среднее;}
}
\item{
\tiny{несмещённая (исправленная) дисперсия — это случайная величина}
\begin{math}
S^2 = \frac{1}{n-1} \sum\limits_{i=1}^n \left(X_i - \bar{X} \right)^2
\end{math}
}
\end{itemize}
\tiny{Очевидно, }
\begin{math}
S^2 = \frac{n}{n-1} S^2_n
\end{math}
\end{block}
\tiny{Из графиков видно, что оценка среднеквадратичного отклонения распределения дефектов для компании Harpy значительно выше, чем для компании Westeros:}
\includegraphics[width=1\linewidth]{image/9.png}
\end{flushleft}
\end{frame}

\begin{frame}
\begin{flushleft}
\footnotesize{4. Итоговым показателем рассмотрим коэффициент Шарпа:}
\includegraphics[width=1\linewidth]{image/10.png}
\end{flushleft}
\end{frame}

\begin{frame}
\begin{flushleft}
\begin{block}{Коэффициент Шарпа}
\tiny{Показатель эффективности инвестиционного портфеля (актива), который вычисляется как отношение средней премии за риск к среднему отклонению портфеля.}
\end{block}
\tiny{Из графиков видно, что коэффициент Шарпа распределения дефектов для компании Harpy значительно ниже, чем для компании Westeros:}
\includegraphics[width=1\linewidth]{image/11.png}
\end{flushleft}
\end{frame}

\begin{frame}
\begin{flushleft}
\frametitle{Вывод}
\small{Таким образом, компания Westeros лидирует по всем трем оценкам, причем столь большая разница по оценке среднеквадратичного отклонения и коэффициенту Шарпа с довольно большой вероятностью не приблизится к 0 за 11 месяцев. Поэтому по результатам разведывательного анализа следует выбрать компанию Westeros.}
\end{flushleft}
\end{frame}

\end{document} 